\documentclass[12pt]{article}
\usepackage{amsmath}
\begin{document}
\center\textbf{Homework 6}\\

\begin{flushleft}
Melanie Rubalcaba\\
12/10/2017

\vspace{0.25cm}

\emph{Question 1}\\
\textbf{(a)}  We will use the product rule:\\
18\(\cdot\)325 = 5850\\
There are 5850 ways we could choose our representatives.\\

\vspace{0.25cm}

\textbf{(b)}  We will  use the sum rule.\\
18 + 325 = 343\\
There are 343 ways we could choose a representative.\\

\vspace{0.25cm}
\emph{Question 5}We will use the product rule.\\
6\(\cdot\)7 = 42.\\
There are 42 ways to choose airlines.\\

\vspace{0.25cm}

\emph{Question 7}\\

We will use the product rule.\\
26\(\cdot\)26\(\cdot\)26 = 17576\\
There are 17576 three letter combinations.\\

\vspace{0.25cm}

\emph{Question 8}\\

We will use the product rule.\\
26\(\cdot\)25\(\cdot\)24 = 15600\\
There are 15600 unique three letter combinations.\\

\vspace{0.25cm}

\emph{Question 9}\\
We will use the product rule.\\
26\(\cdot\)26 = 676.\\
If the first letter must be A, there are 676 three letter combinations.\\

\vspace{0.25cm}

\emph{Question 14}\\
If each bit can only possibly be a 1 or a 0, then the length is n = 2\(^{n}\). Since two positions will be fixed, the equation is altered to become n = 2\(^{n-2}\).\\

\vspace{0.25cm}

\emph{Question 46}\\
\textbf{(a)}  The bride can be included in 6 ways.\\
The number of combinations of 5 from the available 9 is \(\binom{9}{5}\) = 9\(\cdot\)8\(\cdot\)7\(\cdot\)6\(\cdot\)5 = 15120. \\
Apply Product Rule:\\
6\(\cdot\)15120 = 90720\\
If the bride must be in the picture, there are 90720 combinations.\\

\vspace{0.25cm}

\textbf{(b)}  The bride and groom can be included in 30 different ways( Product rule: 6\(\cdot\)5 = 30))\\
The number of combinations of 4 from the available 9 is \(\binom{8}{4}\) = 8\(\cdot\)7\(\cdot\)6\(\cdot\)5 = 1680.\\
Apply Product Rule:\\
30\(\cdot\)1680 = 50400\\
If the bride and groom must be in the picture, there are 50400 combinations.\\

\vspace{0.25cm}

\textbf{(c)}  The bride or the groom can be included in 12 ways (Product Rule: 2\(\cdot\)6 = 12)\\
The number of combinations of 5 from the available 9 is \(\binom{9}{5}\) = 9\(\cdot\)8\(\cdot\)7\(\cdot\)6\(\cdot\)5 = 15120.\\
Apply Product Rule:\\
12\(\cdot\)15120 = 181440\\
If one of the bride and groom must be in the picture, there are 181440 combinations
\end{flushleft}
\end{document}



