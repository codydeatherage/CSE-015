\documentclass{article}

\usepackage{booktabs}

\begin{document}
\center\textbf{HOMEWORK 2}\par

\begin{flushleft}

\textit{Page 78, 1}\par
If Socrates is human, then Socrates is mortal = P\(\rightarrow\) Q\par
Socrates is human. : Socrates is mortal = P:Q\par
The argument form is the valid form of modus ponens.
\end{flushleft}


\begin{flushleft}
\textit{Page 78, 2}\par

If George does not have eight legs, then he is not a spider = P\(\rightarrow\) Q\par
George is a spider. : George has eight legs. = Q : P\par
The argument form is the valid form of modus tollens.

\end{flushleft}

\begin{flushleft}
\textit{Page 78, 5}\par
W : "Randy works hard" \par
D : "Randy is a dull boy" \par
J  : "Randy gets the job"\par
\vspace{0.25cm}
If Randy works hard, then he is a dull boy= W\(\rightarrow\) D \par
If Randy is a dull boy, then he will not get the job =  D\(\rightarrow\lnot\)J\par
\vspace{0.25cm}
Proof:

(1) W\hspace{2cm}Hypothesis\par
(2) W\(\rightarrow\)D\hspace{1.4cm}Hypothesis\par
(3) D\hspace{2.1cm}Modus ponens using (2) and (3)\par
(4) D\(\rightarrow\lnot\)J \hspace{1.15cm}Given\par
(5) \(\lnot\) J \hspace{1.75cm}Modus ponens using (3) and (4)

\end{flushleft}

\begin{flushleft}
\textit{Page 78, 6}\par

Let: \par
R : "It rains"\par
F : "It is foggy"\par
S : "The sailing race will be held"\par
L: : "The life saving demonstration will go on"\par
T : "The trophy will be awarded"\par
\vspace{0.25cm}
If it does not rain or if it is not foggy, then the sailing race will be held and the life-saving demonstration will go on =
(\(\lnot\)R\(\lor\lnot\)F)\(\rightarrow\)(S\(\land\)L)\par
If the sailing race is held, then the trophy will be awarded =  S\(\rightarrow\)T\par
The trophy was not awarded, becomes \(\lnot\)T\par
\vspace{0.25cm}

Proof:\par
(1) \(\lnot\)T\hspace{3.5cm}Hypothesis\par
(2) P\(\rightarrow\)T\hspace{3.15cm}Hypothesis\par
(3) \(\lnot\)P\hspace{3.5cm}Modus tollens using (1) and (2)\par
(4) (\(\lnot\)R\(\lor\lnot\)F)\(\rightarrow\)(P\(\land\)Q)\hspace{1.15cm}Hypothesis\par
(5) (\(\lnot\)(P\(\land\)Q))\(\rightarrow\lnot\)(\(\lnot\)R\(\lor\lnot\)F)\hspace{0.3cm}Contrapositive of (4)\par
(6) (\(\lnot\)P\(\lor\lnot\)Q)\(\rightarrow\)(R\(\land\)F)\hspace{1.15cm}De Morgan's\par
(7) \(\lnot\)P\(\lor\lnot\)Q\hspace{2.75cm}Addition with (3)\par
(8) R\(\land\)F\hspace{3.25cm}Modus ponens using (6) and (7)\par
(9) R\hspace{3.75cm}Simplification of (8)
\end{flushleft}

\begin{flushleft}
\textit{Page 91, 1}\par
Let a and b be two odd integers.\par
a = 2x + 1 and b = 2y + 1, where x and y are integers.\par

 a + b = 2x+1 + 2y + 1 = 2x + 2y + 2\par

2(x + y + 2) must be an even number.

Therefore, the sum of any two odd integers must yield an even result.\par
\vspace{0.25cm}
\textit{Page 91, 2}\par
Let a and b be two even integers.\par
a = 2x and b = 2y, where x and y are integers,\par

 a+ b = 2x + 2y = 2(x + y) which must be even.

Therefore, the sum of any two even integers must yield an even result.\par
\vspace{0.25cm}
\textit{Page 91, 3}\par
Let x be an even integer.\par
x = 2y, where y is an integer,\par
$x^2$ = $(2y)^2$ = 4$y^2$\par
Simplify:\par
2($2y^2$)
\par
This operation yielded an even result.
Therefore, the square of any even number must yield an even result.\par

\vspace{0.25cm}
\textit{Page 91, 6}\par
Let a and b be two odd integers.\par
a = 2x + 1 and n = 2y + 1, where x and y are integers, \par
a \(\cdot\)b = (2x+1)\(\cdot\)(2y+1) = 4xy + 2x + 2y + 1\par
Simplify:\par
2(2xy + x + y) + 1\par
This operation yielded an odd result.\par
Therefore, the product of two odd integers must yield an odd result.\par

\vspace{0.25cm}
\textit{Page 91, 9}\par
Let a and b be rational numbers.\par
Let x be an irrational number.\par
Let a = b + x\par 
Subtract a from both sides of the equation: \par
a + (-b) = b + x + (-b)\par
a + (-b) = x\par
The result concluded that the sum of two rational numbers yielded an irrational number.\par
Therefore, by contradiction, the sum of a rational number and an irrational number must yield an irrational result.\par

\vspace{0.25cm}
\textit{Page 91, 11}\par
Let a = \(\sqrt{5}\).\par
$a^2$ =  \(\sqrt{5}\)$^2$ = 5.\par
This operation squared an irrational number and yielded a rational number.\par
Therefore, the product of two irrational numbers must not always yield an irrational result.\par

\vspace{0.25cm}
\textit{Page 91, 17}\par
\textit{(a)}\par
Let n be  an odd number.\par
n = 2x + 1, where x is an integer,\par
Substitution: $(2x + 1)^3$ + 5 = 8$x^3$ + 12$x^2$ + 6x + 6\par
Simplify:\par
 2(4$x^3$ + 6$x^2$ + 3x + 3).\par
Therefore, $n^3$ + 5 must be even since 2 is a factor.

\vspace{0.25cm}
\textit{(b)}\par
Let n be an odd number.\par
Let $n^3$ + 5 also be an odd number.\par
Since n is odd, and the product of odd numbers is must yield an odd result, $n^3$ must also be an odd number.\par
($n^3$ + 5) - $n^3$ is an odd number subtracted from an odd number which yields an odd result.\par
However, it known that the difference between two odd numbers should yield an even result.\par
Therefore, if $n^3$ + 5 yields an odd result, n must be an even number.

\vspace{0.25cm}
\textit{Page 91, 18}\par
\textit{(a)}\par
Let n be an odd number.\par
Then, n = 2x + 1, where x is an integer.\par
Substitution: 3(2x + 1) + 2 = 6x + 5 = 6x + 4 + 1\par
Simplify:\par
2(3x + 2) + 1\par
This operation yields an odd number when an odd number is substituted in.\par
Therefore, if 3n+2  yields an even result, n must also be even.

\vspace{0.25cm}
\textit{(b)}\par
Let n be an  odd number.\par
Let 3n + 2 be an even number..\par
Since the product of two odd numbers must yield an odd result, then 3n must be an odd number.\par
Therefore, 3n + 2 must also be an odd number.\par
Therefore, by contradiction, if 3n + 2 yields an even result, then n must also be even.

\end{flushleft}
\end{document}
